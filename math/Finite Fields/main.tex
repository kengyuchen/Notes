%%%%%%%%%%%%%%%%%%%%

% Project Name: Finite Fields
% File: main.tex
% Author: Keng-Yu Chen

%%%%%%%%%%%%%%%%%%%%

%%%%%%%%%%%%%%%%%%%%

% Project Name: SVD
% File: header.tex
% Author: Keng-Yu Chen

%%%%%%%%%%%%%%%%%%%%

\documentclass[a4paper]{article}
\usepackage[a4paper, total={6in, 10in}]{geometry}
% \usepackage{showframe}
\usepackage{mathrsfs}
\usepackage{color}
\usepackage[utf8]{inputenc}
\usepackage[T1]{fontenc}
\usepackage{amsthm,amsmath,amssymb}
\usepackage{bm}
\usepackage{bbm}
\usepackage{stmaryrd}
\usepackage{graphicx}
\usepackage{multicol}
\usepackage{multirow}
\usepackage{booktabs}
\usepackage{makecell}
\usepackage{wrapfig}
\usepackage{subcaption}
\usepackage{thmtools}
\usepackage{fancyhdr}
\usepackage{hyperref}
\hypersetup{colorlinks=true, linkcolor=red,citecolor=blue, filecolor=magenta, urlcolor=blue}

\usepackage{algorithm}
\usepackage{algpseudocode}
\renewcommand{\algorithmicrequire}{\textbf{Input:}}
\renewcommand{\algorithmicensure}{\textbf{Output:}}

\usepackage[
    lambda,
    advantage,
    operators,
    sets,
    adversary,
    landau,
    probability,
    notions,
    logic,
    ff,
    mm,
    primitives,
    events,
    complexity,
    oracles,
    asymptotics,
    keys
]{cryptocode}

% Set Self-defined Words
\renewcommand{\verify}{\mathsf{Vrfy}}
\newcommand{\Adv}{\mathbf{Adv}}
\newcommand{\R}{\mathbb{R}}
\newcommand{\N}{\mathbb{N}}
\newcommand{\Z}{\mathbb{Z}}
\newcommand{\Q}{\mathbb{Q}}

\makeatletter
\newcommand{\vast}{\bBigg@{4}}
\newcommand{\Vast}{\bBigg@{5}}
\makeatother

% Convert block[name] into a theorem-like format
% \newtheorem{theorem}{Theorem}

% \theoremstyle{definition}
% \newtheorem{definition}{Definition}

% \theoremstyle{definition}
% \newtheorem{assumption}{Assumption}
% for llncs:
% \spnewtheorem{assumption}{Assumption}{\bfseries}{\itshape}

% \theoremstyle{plain}
% \newtheorem{corollary}{Corollary}

% Add Referemces
\usepackage[
	backend=biber,
	style=alphabetic,
	sorting=ynt
]{biblatex}
\addbibresource{reference.bib}

\title{\textbf{Singular Value Decomposition}}
\author{Keng-Yu Chen}
\date{June 9, 2025}


\begin{document}
\maketitle

We show the classification of finite fields and how the operations of them work.

\section{Classification}

We here show that all finite fields of the same size are isomorphic to the splitting field for the polynomial $X^{p^n} - X$ over $\Z_p$ for some prime $p$ and natural number $n$.

\begin{lemma}
\label{lem-1}
Every finite field has a prime subfield, generated by the unity $1$, of cardinality a prime number $p$, which is its characteristic. The prime subfield is isomorphic to $\Z_p$.

\end{lemma}

\begin{theorem}
\label{thm-1}
Every finite field has $p^n$ elements for some prime $p$ and natural number $n$.
\end{theorem}

\begin{proof}

Let $K$ be any finite field and $F = \Z_p$ be its prime subfield. We can view $K$ as a vector space over the field $F$. Since $K$ is finite, there exists a basis $\{b_1,\cdots,b_n\}$ of $K$. Every element of $K$ can be written uniquely as
\[
	a_1b_1 + \cdots + a_nb_n
\]
where $a_i \in \Z_p$, which implies $|K| = p^n$

\end{proof}

\begin{theorem}
\label{thm-2}
	If a field $K$ has $p^n$ elements, then it is the splitting field for the polynomial $f(X) = X^{p^n} - X$ over $\Z_p$
\end{theorem}

\begin{proof}

Consider the multiplicative group $K^*$, with $p^n - 1$ elements. By the Lagrange theorem, we know every element $x$ of $K$ satisfies $x^{p^n-1} - 1 = 0$. This also means $x^{p^n} - x = 0$. With $0$, these are $p^n$ roots of $f(X)$. But $f(X)$ has exactly $p^n$ roots. Hence the field $K$ is the set of all roots of $f(X)$. $f(X)$ splits over $K$, and $f(X)$ cannot split over any proper subfield.

\end{proof}

With Theorem \ref{thm-1} and Theorem \ref{thm-2}, we know if a field with $p^n$ elements really exists, it is the splitting field over $\Z_p$. Then we show a splitting field really has $p^n$ elements.


\begin{lemma}
\label{lem-2}
If a characteristic of a commutative ring $R$ is a prime $p$, then the map
\[
	\phi: R \to R, \quad \phi(x) = x^p
\]
is an endomorphism of $R$.

\end{lemma}
Note that this implies $(x+y)^p = x^p + y^p$ in $R$.


\begin{theorem}
\label{thm-3}
	The splitting field $K$ for the polynomial $f(X) = X^{p^n} - X$ over $\Z_p$ has $p^n$ elements.
\end{theorem}

\begin{proof}

Let $F$ be the set of all the roots of $f(X)$. By Lemma \ref{lem-2},
\[
	F = \{x \in K \mid x^{p^n} = x\} = \{x \in K \mid \phi(x)^n = x\}
\]
Note that $1 \in F$ and as $\phi$ is an endomorphism, $\phi^n:=\psi$ is also an endomorphism. So for any $a,b \in F$,
\begin{gather*}
	\psi(a+b) = \psi(a)+\psi(b) = a + b \\
	\psi(ab) = \psi(a)\psi(b) = ab \\
	\psi(a^{-1}) = \psi(a)^{-1} = a^{-1}
\end{gather*}
This says $F$ is itself a field. This rather implies $F = K$ is the splitting field for $f(X)$. As there are $p^n$ roots of $f(X)$, it is left to show that all roots are simple.

Given any root $a \neq 0$, we know $a^{p^n - 1} = 1$ and
\[
	f(X) = (X^{p^n - 1} - 1) X = (X^{p^n - 1} - a^{p^n - 1}) X = (X-a)g(X)
\]
for some $g(X)$. Dividing the polynomial,
\[
	g(X) = X^{p^n-1} + aX^{p^n-2} + \cdots + a^{p^n-2}X, \quad g(a) = (p^n-1)a^{p^n - 1} = -1
\]
Hence $a$ is a simple root.

\end{proof}

With Theorem \ref{thm-1}, Theorem \ref{thm-2}, and Theorem \ref{thm-3}, any finite field with cardinality $p^n$ is a splitting field and exists for all prime $p$ and natural number $n$, and they are all sorts of finite fields. Therefore, we then use $\mathbb{F}_{p^n}$ to denote any field with $p^n$ elements. Also, by the proof of Theorem \ref{thm-3}, $\mathbb{F}_{p^n}$ is the set of all the roots of $f(X) = X^{p^n} - X$ over $\Z_p$, and each root is simple.


\section{Operation}

\begin{lemma}
\label{lem-3}
For a finite field $K$, $K^*$ is a cyclic group.
\end{lemma}

\begin{proof}

Firstly, $K^*$ is a finite Abelian group, so we may write $K^*$  as
\[
	K^* = \Z_{p_1^{k_1}} \oplus \cdots \oplus \Z_{p_t^{k_t}}
\]
for some prime powers $p_i^{k_i}$.

Suppose $K^*$ is not a cyclic group, there exist some $p_i$ and $p_j$ that are not coprime, which means $p_i = p_j := p$. (If $m,n$ are coprime, then $\Z_{mn} = \Z_m \oplus \Z_n$ )

$K^*$ thus have a subgroup isomorphic to $\Z_p \oplus \Z_p$. This subgroup contains $p^2 - 1$ elements of order $p$, which means there are $p^2 - 1$ roots to the polynomials $X^p - 1$ over $K$. This contradicts the fact that $X^p-1$ have at most $p$ roots.

\end{proof}


\begin{theorem}
\label{thm-4}
For every $n$, there exists an irreducible polynomial $q(X)$ over $\Z_p$ of degree $n$. Moreover,
\[
	\mathbb{F}_{p^n} = \Z_p[X] / (q(X))
\]
and that $q(X)$ divides $X^{p^n} - X$.

\end{theorem}

\begin{proof}

Let $a$ be the generator of $\mathbb{F}_{p^n}^*$. Then $\mathbb{F}_{p^n} = \Z_p(a)$ since $\mathbb{F}_{p^n}$ contain $\Z_p$ and $a$, and all elements of $\mathbb{F}_{p^n}^*$ are powers of $a$. Now let $q(X)$ be the minimal polynomial of $a$ over $\Z_p$, then
\[
	\text{deg}(q(X)) = [\Z_p(a): \Z_p] = [\mathbb{F}_{p^n}:\Z_p] = n
\]
Moreover, by the isomorphism theorem, since $q(X)$ is irreducible,
\[
	\Z_p(a) = \Z_p[a] = \Z_p[X] / (q(X))
\]

For any root $b$ of $q(X)$ (in some algebraic closure of $\Z_p$). Since $q(X)$ is irreducible, $q(X)$ is also the minimal polynomial of $b$ and thus
\[
	\Z_p[X] / (q(X)) = \Z_p(b) = \mathbb{F}_{p^n}
\]
The field $\mathbb{F}_{p^n}$ is the set of all the roots of $f(X) = X^{p^n} - X$; as a result, $b$ is also a root of $f(X)$.
As all the roots of $q(X)$ in some algebraic closure is a root of $f(X)$, $q(X) | f(X)$.

\end{proof}

Theorem \ref{thm-4} implies that to consider operating on the finite field $\mathbb{F}_{p^n}$, we can first find an irreducible polynomial $q(X)$ of degree $n$ (which must exist), and then consider operating on the field $\Z_p[X] / (q(X))$. Moreover, by the following theorem, such $q(X)$ divides $X^{p^n} - X$.

\begin{theorem}
\label{thm-5}
Let $q(X)$ be an irreducible polynomial over $\Z_p$ of degree $d |n$, then $q(X)$ divides $X^{p^n} - X$.
\end{theorem}

\begin{proof}

Let $a$ be a root of $q(X)$ in some extension of $\Z_p$. As $q(X)$ is irreducible, $q(X)$ is the minimal polynomial of $a$ and thus
\[
	[\Z_p(a) : \Z_p] = \text{deg}(q(X)) = d
\]
But this then implies $\Z_p(a)$ has $p^d$ elements, $\Z_p(a) = \mathbb{Z}_{p^d}$, which further implies
\[
	a^{p^d} = a
\]
But as $d | n$, $p^n = (p^d)^l$ for some $l$, and $a^{p^n} = (a^{p^d})^{p^d} \cdots = a$, we see $a$ is also a root of $X^{p^n} - X$. This implies $q(X)$ divides $X^{p^n} - X$.

\end{proof}

Finally, we show that all irreducible polynomials that divide $X^{p^n} - X$ can be used to construct the finite field $\mathbb{F}_{p^n}$.

\begin{theorem}
\label{thm-6}
$f(X) = X^{p^n} - X$ over $\Z_p$ is the product of all monic irreducible polynomials over $\Z_p$ whose degree $d | n$.

\end{theorem}

\begin{proof}

On the one hand, from Theorem \ref{thm-5}, we already know that any monic irreducible polynomials over $\Z_p$ whose degree $d | n$ divides $f(X)$. Since irreducible polynomials in $\Z_p[X]$ are coprime (otherwise, an element can have two minimal polynomials), we have
\[
	\prod_{\text{monic irr. } q \in \Z_p[X], \text{deg}(q)|n} q(X) \mid f(X)
\]

On the other hand, recall that
\[
	f(X) = \prod_{\alpha \in \mathbb{F}_{p^n}} (X-\alpha)
\]
For each $\alpha \in \mathbb{F}_{p^n}$, its minimal polynomial $q_\alpha(X)$ must be of some degree $d | n$ since
\[
\text{deg}(q_\alpha(X)) = [\Z_p(\alpha):\Z_p], \quad [\Z_p(\alpha):\Z_p]  \mid [\mathbb{F}_{p^n} : \Z_p] = n
\]
Therefore, $X-\alpha$ divides some monic irreducible polynomial whose degree divides $n$. As each $X-\alpha$ is coprime,
\[
	f(X) = \prod_{\alpha \in \mathbb{F}_{p^n}} (X-\alpha) \mid \prod_{\text{monic irr. } q \in \Z_p[X], \text{deg}(q)|n} q(X).
\]
Hence,
\[
	f(X) = \prod_{\text{monic irr. } q \in \Z_p[X], \text{deg}(q)|n} q(X)
\]

\end{proof}


\section{Subfield}

Finally, we discuss subfields of finite fields $\mathbb{F}_{p^n}$. Note that these subfields are also finite fields and have the same characteristic as the original field.

We first show that if $d \mid n$, we have a subfield $\mathbb{F}_{p^d}$.

\begin{theorem}
\label{thm-7}
Let $d \mid n$. The set
\[
	L := \{x \in \mathbb{F}_{p^n} \mid x^{p^d} = x\}
\]
is a subfield of $\mathbb{F}_{p^n}$ and $|L| = p^d$.
\end{theorem}

\begin{proof}

By Lemma \ref{lem-2} and the proof of Theorem \ref{thm-3}, we know that $\phi(x) = x^p$ over $\Z_p$ is an endomorphism and thus $\psi := \phi^d$ is also an endomorphism. This implies $L$ is a field in $\mathbb{F}_{p^n}$. But since each element in $\mathbb{F}_{p^n}$ is distinct, $|L| = \#\{\text{roots of } X^{p^d} - X\} =  p^d$.

\end{proof}

Next, we show that if $\mathbb{F}_{p^d}$ is a subfield, $d \mid n$.

\begin{theorem}
\label{thm-8}
Let $L$ be a subfield of $\mathbb{F}_{p^n}$. Then $|L| = p^d$ for some $d \mid n$.

\end{theorem}

\begin{proof}

Since $L$ is a subfield of $\mathbb{F}_{p^n}$, its characteristic is $p$, and thus $|L| = p^d$ for some natural number $d$. Suppose $d \nmid n$, all elements of $L$ are solutions to both equations
\[
	X^{p^d} - X = 0 \quad \text{and} \quad X^{p^n} - X = 0
\]

Let $a\in L$, and let $n = dq + r$ for some $r < d$, then
\[
	\alpha = \alpha^{p^n} = (\alpha^{p^d})^{p^{n-d}} = \alpha^{p^{n-d}} = \cdots = \alpha^{p^{r}}
\]
Hence, $\alpha$ is also a root of the polynomial of $X^{p^r} - X$. There are at most $p^r$ roots of this polynomial, but there are $|L| = p^d > p^r$ elements in $L$. Since we have shown in previous theorems that the roots of the polynomial $X^{p^d} - X$ are all simple, this fact leads to a contradiction.

\end{proof}


Theorem \ref{thm-7} and Theorem \ref{thm-8} show that every divisor $d$ of $n$ corresponds to a unique subfield $\mathbb{F}_{p^d}$ of $\mathbb{F}_{p^n}$.

We can now view $\mathbb{F}_{p^m}$ and $\mathbb{F}_{p^n}$ as subfields of $\mathbb{F}_{p^{mn}}$. This implies the following theorem.

\begin{theorem}
The set
\[
	\bar \Z_p := \bigcup_{n \in \N} \mathbb{F}_{p^n}
\]
is a field. Moreover, it is the algebraic closure of $\Z_p$.
\end{theorem}


\begin{proof}

\mbox{} \\

% \vspace{4pt}
\noindent \textbf{$\bar \Z_p$ is a field}

Given any $a \neq 0,b \in \bar \Z_p$, there are some natural numbers $m,n$ such that $a \in \mathbb{F}_{p^m}$ and $\mathbb{F}_{p^n}$. Since both fields are subfields of $\mathbb{F}_{p^{mn}}$, $a,b \in \mathbb{F}_{p^{mn}}$, and thus $a\pm b, ab, a^{-1}b$ are all in $\mathbb{F}_{p^{mn}} \subset \bar \Z_p$.

\vspace{4pt}
\noindent \textbf{$\bar \Z_p$ is algebraically closed.}

Given any polynomial $f(X) = \sum_{i=1}^m a_iX^i$ over $\bar \Z_p$, where $a_i \in \mathbb{F}_{p^{n_i}}$, we can view it as a polynomial over $\mathbb{F}_{p^M}$, where $M = l.c.m(\{n_i\})$. Let $g(X)$ be an irreducible polynomial that divides $f(X)$. Such $g(X)$ must exist since $f(X) \in \bar \Z_p[X]$, a principal ideal domain.

Consider the ring $K = \Z_{p^M}[X] /(g(X))$. Note that it is an extension of $\Z_{p^M}$, and $g(X)$ has a root in $K$ since
\[
	g(X + (g(X))) = g(X) + (g(X)) = 0
\]
Moreover, $K$ is finite and contains $(p^M)^{\text{deg}(g(X))}$ elements. All finite fields are of size $q^N$ for some prime number $q$ and natural number $N$, where $q$ is its characteristic. Hence, $q = p$ and $K = \mathbb{F}_{p^{M\text{deg}(g(X))}}$. We thus show that $g(X)$ has a root in $\mathbb{F}_{p^{M\text{deg}(g(X))}} \subset \bar \Z_p$, which implies $f(X)$ also has a root in $\bar\Z_p$.

\vspace{4pt}
\noindent \textbf{$\bar \Z_p$ is an algebraic extension of $\Z_p$.}

Given any $a \in \bar \Z_p$, there is some natural numbers $n$ such that $a \in \mathbb{F}_{p^n}$, which implies that $a$ is a root of the polynomial $X^{p^n} - X$ over $\Z_p$.

\end{proof}



%-------------------
%% Reference List
\pagebreak

\nocite{*}
\printbibliography

\end{document}
