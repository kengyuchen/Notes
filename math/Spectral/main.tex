%%%%%%%%%%%%%%%%%%%%

% Project Name: Spectral Theorem
% File: main.tex
% Author: Keng-Yu Chen

%%%%%%%%%%%%%%%%%%%%

%%%%%%%%%%%%%%%%%%%%

% Project Name: PCA
% File: header.tex
% Author: Keng-Yu Chen

%%%%%%%%%%%%%%%%%%%%

\documentclass[a4paper]{article}
\usepackage[a4paper, total={6in, 10in}]{geometry}
% \usepackage{showframe}
\usepackage{mathrsfs}
\usepackage{color}
\usepackage[utf8]{inputenc}
\usepackage[T1]{fontenc}
\usepackage{amsthm,amsmath,amssymb}
\usepackage{bm}
\usepackage{bbm}
\usepackage{stmaryrd}
\usepackage{graphicx}
\usepackage{multicol}
\usepackage{multirow}
\usepackage{booktabs}
\usepackage{makecell}
\usepackage{wrapfig}
\usepackage{subcaption}
\usepackage{thmtools}
\usepackage{fancyhdr}
\usepackage{hyperref}
\hypersetup{colorlinks=true, linkcolor=red,citecolor=blue, filecolor=magenta, urlcolor=blue}

\usepackage{algorithm}
\usepackage{algpseudocode}
\renewcommand{\algorithmicrequire}{\textbf{Input:}}
\renewcommand{\algorithmicensure}{\textbf{Output:}}

\usepackage[
    lambda,
    advantage,
    operators,
    sets,
    adversary,
    landau,
    probability,
    notions,
    logic,
    ff,
    mm,
    primitives,
    events,
    complexity,
    oracles,
    asymptotics,
    keys
]{cryptocode}

% Set Self-defined Words
\renewcommand{\verify}{\mathsf{Vrfy}}
\newcommand{\Adv}{\mathbf{Adv}}
\newcommand{\R}{\mathbb{R}}
\newcommand{\N}{\mathbb{N}}
\newcommand{\Z}{\mathbb{Z}}
\newcommand{\Q}{\mathbb{Q}}

\makeatletter
\newcommand{\vast}{\bBigg@{4}}
\newcommand{\Vast}{\bBigg@{5}}
\makeatother

% Convert block[name] into a theorem-like format
\newtheorem{theorem}{Theorem}

% \theoremstyle{definition}
% \newtheorem{definition}{Definition}

% \theoremstyle{definition}
% \newtheorem{assumption}{Assumption}
% for llncs:
% \spnewtheorem{assumption}{Assumption}{\bfseries}{\itshape}

% \theoremstyle{plain}
% \newtheorem{corollary}{Corollary}

% Add Referemces
\usepackage[
	backend=biber,
	style=alphabetic,
	sorting=ynt
]{biblatex}
\addbibresource{reference.bib}

\title{\textbf{Principal Component Analysis}}
\author{Keng-Yu Chen}
\date{June 9, 2025}


\begin{document}
\maketitle

\begin{theorem*}[Spectral Theorem]
	Every real symmetric matrix can be diagonalized by an orthonormal basis\footnote{The theorem and the proofs all hold for \emph{self-adjoint} matrices, a more general class of complex matrices.}.
\end{theorem*}

We prove this theorem through the following steps.

In the following, we consider a finite $n$-dimensional vector space $V$ and a real symmetric matrix $A$ for a linear transformation $T$ on $V$.

\begin{theorem}
\label{thm-1}
	Eigenvalues of real symmetric matrices are real.
\end{theorem}

\begin{proof}

Let $A$ be a real symmetric matrix. For any eigenvector $v \in V$ of $A$ with corresponding eigenvalue $\lambda$,
\begin{align*}
	\langle Av,v \rangle &= \langle \lambda v,v \rangle = \lambda \langle v, v \rangle \\
	&=\langle v, A^*v\rangle = \langle v, Av \rangle = \langle v,\lambda v \rangle = \bar \lambda \langle v,v \rangle
\end{align*}
Therefore, $\lambda = \bar \lambda$, $\lambda$ is real.

Note that if $A$ is a real symmetric matrix, then its eigenvectors are also real since they are in the kernel of $A-\lambda I$ for some real $\lambda$.

\end{proof}

\begin{theorem}
	Eigenvectors to distinct eigenvalues of real symmetric matrices are orthogonal
\end{theorem}

\begin{proof}

	Let $v, w$ be two eigenvectors of $A$, associated with real eigenvalues $\lambda,\mu$.
\begin{align*}
	\langle Av,w \rangle &= \lambda \langle v,w \rangle \\
	&= \langle v,A^*w \rangle = \langle v,Aw \rangle = \langle v,\mu w \rangle = \bar \mu \langle v,w \rangle
\end{align*}
Therefore, either $\lambda = \bar \mu = \mu$ or $\langle v,w\rangle = 0$.

\end{proof}


\begin{theorem}
	Let $W$ be a $T$-invariant space in $V$, then its orthogonal complement $W^\perp$ is also a $T$-invariant space.
\end{theorem}

\begin{proof}

For any $x \in W, y \in W^\perp$,
\[
	\langle x, Ay\rangle = \langle A^*x, y \rangle = \langle Ax, y \rangle = \lambda \langle x,y \rangle = 0,
\]
which shows $Ay$ also lies in $W^\perp$.

Note that this implies for any eigenvector $v$ of a real symmetric $A$ and $w \perp v$, $Aw \perp v$.

\end{proof}

Now we can prove the spectral theorem.

\begin{proof}
	
	We prove the theorem by induction. The case $n=1$ is trivial. For a general finite $n$-dimensional vector space $V$, by the \emph{fundamental theorem of algebra}, there must exist a root of the characteristic polynomial, which is an eigenvalue. By Theorem \ref{thm-1}, it is real, so we can find a 1-unit eigenvector $v$ with a real eigenvalue $\lambda$.

Let $W = \text{span}\{v\}$, $W^\perp$ be the orthogonal complement of $W$. Since $W^\perp$ is $n-1$-dimensional, by induction, we can find an orthonormal basis $\mathcal{\tilde B}$ for $W^\perp$. As $W$ and $W^\perp$ are both $T$-invariant, under the basis $\mathcal{B} = \{v\} \cup \mathcal{\tilde B}$, $T$ corresponds to a block matrix,
\[	
	A \sim \begin{bmatrix} 
		\lambda & 0 \\
		0 & \tilde A
	\end{bmatrix},
\]
where $\tilde A$ is the matrix representation of $T$ under the basis $\tilde B$, restricted on the space $W^\perp$, and $\tilde A$ is a diagonal matrix. Moreover, since $W^\perp$ is the orthogonal complement of $W=\{v\}$, $v$ is orthogonal to every vector in $W^\perp$, including all the basis vectors $\mathcal{\tilde B}$. Hence, $\mathcal{B}$ is an orthonormal basis for $V$.

\end{proof}



%-------------------
%% Reference List
\pagebreak

\nocite{*}
\printbibliography

\end{document}
