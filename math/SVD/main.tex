%%%%%%%%%%%%%%%%%%%%

% Project Name: SVD
% File: main.tex
% Author: Keng-Yu Chen

%%%%%%%%%%%%%%%%%%%%

%%%%%%%%%%%%%%%%%%%%

% Project Name: SVD
% File: header.tex
% Author: Keng-Yu Chen

%%%%%%%%%%%%%%%%%%%%

\documentclass[a4paper]{article}
\usepackage[a4paper, total={6in, 10in}]{geometry}
% \usepackage{showframe}
\usepackage{mathrsfs}
\usepackage{color}
\usepackage[utf8]{inputenc}
\usepackage[T1]{fontenc}
\usepackage{amsthm,amsmath,amssymb}
\usepackage{bm}
\usepackage{bbm}
\usepackage{stmaryrd}
\usepackage{graphicx}
\usepackage{multicol}
\usepackage{multirow}
\usepackage{booktabs}
\usepackage{makecell}
\usepackage{wrapfig}
\usepackage{subcaption}
\usepackage{thmtools}
\usepackage{fancyhdr}
\usepackage{hyperref}
\hypersetup{colorlinks=true, linkcolor=red,citecolor=blue, filecolor=magenta, urlcolor=blue}

\usepackage{algorithm}
\usepackage{algpseudocode}
\renewcommand{\algorithmicrequire}{\textbf{Input:}}
\renewcommand{\algorithmicensure}{\textbf{Output:}}

\usepackage[
    lambda,
    advantage,
    operators,
    sets,
    adversary,
    landau,
    probability,
    notions,
    logic,
    ff,
    mm,
    primitives,
    events,
    complexity,
    oracles,
    asymptotics,
    keys
]{cryptocode}

% Set Self-defined Words
\renewcommand{\verify}{\mathsf{Vrfy}}
\newcommand{\Adv}{\mathbf{Adv}}
\newcommand{\R}{\mathbb{R}}
\newcommand{\N}{\mathbb{N}}
\newcommand{\Z}{\mathbb{Z}}
\newcommand{\Q}{\mathbb{Q}}

\makeatletter
\newcommand{\vast}{\bBigg@{4}}
\newcommand{\Vast}{\bBigg@{5}}
\makeatother

% Convert block[name] into a theorem-like format
% \newtheorem{theorem}{Theorem}

% \theoremstyle{definition}
% \newtheorem{definition}{Definition}

% \theoremstyle{definition}
% \newtheorem{assumption}{Assumption}
% for llncs:
% \spnewtheorem{assumption}{Assumption}{\bfseries}{\itshape}

% \theoremstyle{plain}
% \newtheorem{corollary}{Corollary}

% Add Referemces
\usepackage[
	backend=biber,
	style=alphabetic,
	sorting=ynt
]{biblatex}
\addbibresource{reference.bib}

\title{\textbf{Singular Value Decomposition}}
\author{Keng-Yu Chen}
\date{June 9, 2025}


\begin{document}
\maketitle

Singular Value Decomposition (SVD) is a way to decompose any real matrix $A \in \R^{m\times n}$ into a simple form:
\[
	A = U \Sigma V^T,
\]
where $U \in \R^{m \times m}$ and $V \in \R^{n \times n}$ and $\Sigma \in \R^{m \times n}$ is a matrix with only diagonal values. The following proof is based on \cite{svd-note}.

\section{Proof of Existence}

\subsection{Case $n = m$}

Since $A^TA$ is an $n \times n$ positive semi-definite (PSD) matrix, we can diagonalize it as
\[
	A^TA = V \Lambda V^T = \sum_{i=1}^n \lambda_i \mathbf{v}_i \mathbf{v}_i^T = \sum_{i=1}^n \sigma_i^2 \mathbf{v}_i \mathbf{v}_i^T,
\]
where $V$ is an orthonormal matrix with eigenvectors of $A^TA$ as its column vectors, and $\Lambda$ is a diagonal matrix with non-negative eigenvalues. 
\[
	V = [\mathbf{v}_1\, \mathbf{v}_2\, \cdots\, \mathbf{v}_n]
\]
\[
	\Lambda = 
	\begin{bmatrix}
		\lambda_1 &0 &\cdots &0 \\
		0 &\lambda_2 & &\vdots \\
		\vdots & &\ddots \\
		0 &\cdots & &\lambda_n
	\end{bmatrix}
	,\quad \lambda_i = \sigma_i^2 \geq 0
\]
By convention, we permute the eigenvalues in descending order; that is, $\lambda_1 \geq \lambda_2 \geq \cdots \geq \lambda_n$.

Now, consider the following vectors with $m$-dimension:
\[
	\mathbf{u}_i = \frac{A \mathbf{v}_i}{\sigma_i}, \quad 1\leq i \leq m
\]
One can check each $\mathbf{u}_i$ is a unit vector
\[
	\|\mathbf{u}_i\|^2 = \frac{\mathbf{v}_i^TA^TA\mathbf{v}_i}{\sigma_i^2} = \frac{\mathbf{v}_i^T (\sigma_i^2 \mathbf{v}_i)}{\sigma_i^2} = 1
\]
If we define the following matrix
\[
	U = [\mathbf{u}_1\, \mathbf{u}_2\, \cdots\, \mathbf{u}_m]
\]
\[
	\Sigma = 
	\begin{bmatrix}
		\sigma_1 &0 &\cdots &0 \\
		0 &\sigma_2 & &\vdots \\
		\vdots & &\ddots \\
		0 &\cdots & &\sigma_n
	\end{bmatrix}
\]
We see
\[
	U = AV\Sigma^{-1} \implies A = U\Sigma V^{-1} = U\Sigma V^T
\]

\subsection{Case $n \neq m$}

If $n \geq m$, since $\sigma_i = 0$ for all $i >m$,  we may simply remove the last $n-m$ columns in $\Sigma$. The equation above becomes
\[
	U = AV
	\begin{bmatrix}
		\frac{1}{\sigma_1} &0 &\cdots &0 \\
		0 &\frac{1}{\sigma_2} & & \vdots \\
		\vdots & &\ddots \\
		0 & & &\frac{1}{\sigma_m} \\
		\vdots & & & \vdots\\
		0 &\cdots & &0
	\end{bmatrix}
	\implies A = U 
	\begin{bmatrix}
		\sigma_1 &0 &\cdots & & &0 \\
		0 &\sigma_2 & & & &\vdots \\
		\vdots & &\ddots \\
		0 &\cdots & &\sigma_m &\cdots &0
	\end{bmatrix}
	V^T
\]
If $n < m$, we can remove the last $m-n$ rows in $\Sigma$, leading to
\[
	U = AV
	\begin{bmatrix}
		\frac{1}{\sigma_1} &0 &\cdots & & &0 \\
		0 &\frac{1}{\sigma_2} & & & &\vdots \\
		\vdots & &\ddots \\
		0 &\cdots & &\frac{1}{\sigma_n} &\cdots &0
	\end{bmatrix}
	\implies A = U
	\begin{bmatrix}
		\sigma_1 &0 &\cdots &0 \\
		0 &\sigma_2 & & \vdots \\
		\vdots & &\ddots \\
		0 & & &\sigma_n \\
		\vdots & & & \vdots\\
		0 &\cdots & &0
	\end{bmatrix}
	 V^T
\]
Note that the vectors $\{\mathbf{u}_i\}_{i > m}$ are not defined. We replace those undefined vectors by any $n-m$ independent vectors which makes $\{\mathbf{u}_i\}$ span the whole $\mathbb{R}^m$.

%-------------------
%% Reference List
\pagebreak

\nocite{*}
\printbibliography

\end{document}
