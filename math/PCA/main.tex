%%%%%%%%%%%%%%%%%%%%

% Project Name: PCA
% File: main.tex
% Author: Keng-Yu Chen

%%%%%%%%%%%%%%%%%%%%

%%%%%%%%%%%%%%%%%%%%

% Project Name: SVD
% File: header.tex
% Author: Keng-Yu Chen

%%%%%%%%%%%%%%%%%%%%

\documentclass[a4paper]{article}
\usepackage[a4paper, total={6in, 10in}]{geometry}
% \usepackage{showframe}
\usepackage{mathrsfs}
\usepackage{color}
\usepackage[utf8]{inputenc}
\usepackage[T1]{fontenc}
\usepackage{amsthm,amsmath,amssymb}
\usepackage{bm}
\usepackage{bbm}
\usepackage{stmaryrd}
\usepackage{graphicx}
\usepackage{multicol}
\usepackage{multirow}
\usepackage{booktabs}
\usepackage{makecell}
\usepackage{wrapfig}
\usepackage{subcaption}
\usepackage{thmtools}
\usepackage{fancyhdr}
\usepackage{hyperref}
\hypersetup{colorlinks=true, linkcolor=red,citecolor=blue, filecolor=magenta, urlcolor=blue}

\usepackage{algorithm}
\usepackage{algpseudocode}
\renewcommand{\algorithmicrequire}{\textbf{Input:}}
\renewcommand{\algorithmicensure}{\textbf{Output:}}

\usepackage[
    lambda,
    advantage,
    operators,
    sets,
    adversary,
    landau,
    probability,
    notions,
    logic,
    ff,
    mm,
    primitives,
    events,
    complexity,
    oracles,
    asymptotics,
    keys
]{cryptocode}

% Set Self-defined Words
\renewcommand{\verify}{\mathsf{Vrfy}}
\newcommand{\Adv}{\mathbf{Adv}}
\newcommand{\R}{\mathbb{R}}
\newcommand{\N}{\mathbb{N}}
\newcommand{\Z}{\mathbb{Z}}
\newcommand{\Q}{\mathbb{Q}}

\makeatletter
\newcommand{\vast}{\bBigg@{4}}
\newcommand{\Vast}{\bBigg@{5}}
\makeatother

% Convert block[name] into a theorem-like format
% \newtheorem{theorem}{Theorem}

% \theoremstyle{definition}
% \newtheorem{definition}{Definition}

% \theoremstyle{definition}
% \newtheorem{assumption}{Assumption}
% for llncs:
% \spnewtheorem{assumption}{Assumption}{\bfseries}{\itshape}

% \theoremstyle{plain}
% \newtheorem{corollary}{Corollary}

% Add Referemces
\usepackage[
	backend=biber,
	style=alphabetic,
	sorting=ynt
]{biblatex}
\addbibresource{reference.bib}

\title{\textbf{Singular Value Decomposition}}
\author{Keng-Yu Chen}
\date{June 9, 2025}


\begin{document}
\maketitle

Principal Component Analysis (PCA) can be viewed as a linear transform which maps a $n$-dimensional random vector to a smaller dimensional vector, but preserving most of its information.

In the following we consider a random vector $x=(x_1, x_2, \dots,\; x_n)^T$, where each $x_i$ is a random variable, representing one attribute or property of $x$. Inner product of two vectors $x, y$ is written as $x^Ty$. All vectors are column vectors.

\section{Covariance Matrix}

The variance of a random variable $x_i$ is defined like
\[
	Var(x_i) = \mathbb{E}(x_i^2) - \mathbb{E}(x_i)^2
\]
where $\mathbb{E}(\cdot)$ is the expectation function. To define the variance of a random vector (a series of random variable), we use the covariance matrix.

The covariance (matrix) of a random vector $x$ can be defined as
\[
	[Cov(x)]_{ij} = Cov(x_i, x_j) =  \mathbb{E}[x_ix_j] - \mathbb{E}[x_i]\mathbb{E}[x_j]
\]


\begin{theorem}
\label{thm-1}
Let $v$ be any constant $m$-dimension vector, the statement holds
\[
	Var(v^Tx)=v^T Cov(x) v
\]
\end{theorem}

\begin{proof}

\begin{align*}
	Var(v^Tx)
	&= Cov(v^Tx, v^Tx) \\
	&= Cov(v_1x_1+\cdots+v_nx_n, v^Tx) \\
	&= v^T\left( Cov(x_i, v^Tx) \right)_{i\leq n} \\
	&= v^T\left( Cov(x_i, v_1x_1+\cdots+v_nx_n) \right)_{i\leq n} \\
	&= v^T Cov(x) v
\end{align*}

\end{proof}

\begin{theorem}
\label{thm-2}
Let $\{\alpha_k\}_n$ be a series of vectors such that
\[
	\alpha_k = \argmax_v \left\{ Var(v^Tx) \mid \|v\|=1, v^Tx \perp \{\alpha_i^Tx\}_{i<k} \right\}
\]
Then $\alpha_k$ is the eigenvector corresponding to the $k$-th largest eigenvalues for matrix $Cov(x)$.

\end{theorem}

\begin{proof}
	Let $\Sigma=Cov(x)$. First consider case $k=1$. We use the Lagrange multiplier
\[
	L=Var(v^Tx)-\beta(v^Tv - 1)=v^T\Sigma v - \beta(v^Tv - 1)
\]
By differentiation,
\[
	\frac{\partial}{\partial v}L = 2 \Sigma v - 2 \beta v = 0 \Longleftrightarrow \Sigma v = \beta v
\]
We see when the maximum is attained, the vector is the eigenvector of $\Sigma$ that corresponds to the largest eigenvalue. We derive the first vector $\alpha_1$.

For $k=2$, we want $\alpha_2^Tx \perp \alpha_1^Tx$, and the condition is equivalent to
\[
	0 = Cov(\alpha_2^Tx, \alpha_1^Tx) = \alpha_2^T \Sigma \alpha_1 = \beta \alpha_2^T \alpha_1
\]
So we just require that $\alpha_2^T \alpha_1 = 0$. Again, by the Lagrange multiplier,
\[
	L=Var(v^Tx)-\beta_1(v^Tv - 1) - \beta_2(v^T\alpha_1) = v^T\Sigma v - \beta(v^Tv - 1) - \beta_2(v^T\alpha_1)
\]
By differentiation,
\[
	\frac{\partial}{\partial v}L = 2 \Sigma v - 2 \beta_1 v - \beta_2 \alpha_1 = 0
\]
Since $\alpha_2 \perp \alpha_1$, the value for $\beta_2 = 0$ naturally, and we obtain
\[
\Sigma v = \beta_1 v
\]
When the maximum is attained, the vector is the eigenvector of $\Sigma$ that corresponds to the second largest eigenvalue. This gives us $\alpha_2$.

The remaining vectors can be derived in a similar way.

\end{proof}


\section{Principal Component}
So far we get a series of independent vectors $\alpha_k$ which, by Theorem \ref{thm-2}, maximize the variance of $v^Tx$. The first component of $x$ can be defined as $\alpha_1^Tx$, and the second component is defined as $\alpha_2^Tx$, and so on. We may think it as a series of linear transformations with corresponding matrices $A_i$ such that 
\[
	A_1 = [\alpha_1], \; A_2 = [\alpha_1 \; \alpha_2], \; \cdots, \; A_n = [\alpha_1 \; \alpha_2 \; \cdots \alpha_n]
\]

If we want to reduce the vector $x$ to a smaller dimension $r$ by a linear transformation, which is what PCA does, we simply do the corresponding matrix multiplication as
\[
	y = A_r^T x  = (\alpha_1^T x,\; \alpha_2^T x,\; \cdots,\; \alpha_r^T x)
\]
Notice that the variance is maximized in the scope of linear transformation.


%-------------------
%% Reference List
\pagebreak

\nocite{*}
\printbibliography

\end{document}
